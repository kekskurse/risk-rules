\documentclass{article}
\usepackage[utf8]{inputenc}
\usepackage{latexsym}
\usepackage[paper=a4paper,left=25mm,right=25mm,top=25mm,bottom=25mm]{geometry}

\title{Risk Rules}
\date{\today}
\author{risk.kekskurse.de}
\begin{document}

\maketitle

\section{Beginn of the Game}

\subsection{Start Territory}
$\Box$ The starting player is selected by the roll of the dices. After this, each player can choose their terretories, one per round\\
$\Box$ The Cards are mixed at and give to the Player, each Player put one armie in each of the territorys
\subsection{Start Armies}
$\Box$ There are no start armies, the first round starts with the reinforcment\\
$\Box$ The number of the start armies are \_\_\_\_\_\_\_\_, in the first round nobody gets a reinforcment\\
$\Box$ The number of the start armies are \_\_\_\_\_\_\_\_, in the first round only the last Player gets a reinforcment\\
$\Box$ The number of the start armies are \_\_\_\_\_\_\_\_, in the first round only the two last Player get a reinforcment\\

\section{Beginning of the Round}

\subsection{Reinforcement}
$\Box$ Number of territory divided by 3 + Armes per Continent + Bonus for the Cards.\\

\subsection{Bonus of the Cards}
\textit{In the new Version of the Game the number of the extra Armys is predefined by the Stars on the Border. }\\
$\Box$ The number of armies based only on the Numbers of the stars\\

\textit{In the old Versions}\\
$\Box$ If you have 3 cards of the same typ (person, rider, cannon) or 3 different typs you get 7 armies\\
$\Box$ If you have 3 cards of the person you get 6 armies, if you have 3 cards of the rider you get 7 armies, if you have 3 cards of the cannon you get 8 armies, if you have 3 different types you get 10 armies\\

\textit{In both Versionen}\\
$\Box$ If the territory of the card is owned by the Player who use the card, the Player gets 2 more armies which can placed in any territory of the Player\\
$\Box$ If the territory of the card is owned by the Player who uses the card, the Player gets 2 more armies which are placed in the territory on the card.

\section{Attack}

\subsection{Attack}
$\Box$ The Person who attacks can decide bevor he/she role the dices how many armys attack, the possible numbers are 1, 2 and 3.

\subsection{Defend}
$\Box$ The Person who deffend can decide bevor the attacker roles the dices how many armies should defend, the possible numbers are 1 and 2\\
$\Box$ The Person who defend can decide after the attacker rols the dices how much armys should defend, the possible numbers are 1 and 2

\subsection{Evaluation of the Dices}
$\Box$ The highest number of the attacker vs. the highest number of the deffender, the second highest again from the attacker vs the second highest from the deffender

\subsection{If the attack win}
$\Box$ The Player can only move the numbers of Armies from the Attack (maximum 3)\\
$\Box$ The Player can move as much armies as he/she wish, the minimum is the numbers from the Attack

\subsection{If the attack win}
$\Box$ The Player can use the Armies after the Attack to make directly a secound attack\\
$\Box$ Each Armie can only attack and move to one territory in one round, no more attacks after it

\section{End of the Round}
\subsection{Movmend}
$\Box$ At the end the Player can move armies from on territory to one neighbouring territory, the Player can do this only one time\\
$\Box$ At the end the Player can move armies from on territory to one neighbouring territory, the Player can do this as often as he/she wants but only once with each army\\
$\Box$ At the end the Player can move armies from on territory to another one as long as there is a land bridge between, the Player can do this only one time.\\
\\
\textit{and for any of this 3 options:}\\\\
$\Box$ The Player can move any armie\\
$\Box$ The Player can only move armies wich was not attacking in this round.

\end{document}
